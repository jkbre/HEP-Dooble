% LaTeX Test Document for Symbol Card Layouts
\documentclass[11pt,a4paper]{article}
\usepackage[margin=1.9cm]{geometry}
\usepackage[utf8]{inputenc}
\usepackage[T1]{fontenc}
\usepackage{graphicx}
\usepackage{amsmath,amssymb}
\usepackage{hyperref}
\usepackage{qrcode}
\usepackage{xcolor}
\usepackage{tabularx} % Required for Idea 1
\usepackage{letltxmacro} % Required for Idea 4

\hypersetup{colorlinks=true,linkcolor=blue,urlcolor=teal}
\pagestyle{plain}
\setlength{\parindent}{0pt}

% --- IDEA 1: Three-Column Layout ---
\newcommand{\symbolcardA}[4]{%
  \noindent\hrule\vspace{1em}
  \begin{tabularx}{\textwidth}{@{} m{4.5cm} X m{2.5cm} @{}}
    % Column 1: Image
    \includegraphics[width=\linewidth]{#1} &
    % Column 2: Title & Description
    {\Large\bfseries #3}\par\medskip#4 &
    % Column 3: QR Code
    \centering\qrcode[height=2.5cm]{#2}
  \end{tabularx}
  \vspace{1em}
}

% --- IDEA 2: Image-Centric with Text Below ---
\newcommand{\symbolcardB}[4]{%
  \noindent\hrule\vspace{1em}
  \begin{center}
    \includegraphics[width=6cm]{#1}
  \end{center}
  \vspace{0.5em}
  \begin{minipage}[t]{0.7\textwidth}
    \raggedright
    {\Large\bfseries #3}\par
    #4
  \end{minipage}%
  \hfill
  \begin{minipage}[t]{0.25\textwidth}
    \raggedleft
    \qrcode{#2}
  \end{minipage}
  \vspace{1em}
}

% --- IDEA 3: Refined Two-Column Layout ---
\newcommand{\symbolcardC}[4]{%
  \noindent\hrule\vspace{1em}
  \begin{minipage}[c]{0.3\textwidth}
    \includegraphics[width=\linewidth]{#1}
  \end{minipage}\hfill
  \begin{minipage}[c]{0.65\textwidth}
    {\Large\bfseries #3}\par
    #4
    \par\vspace{1em}\raggedleft
    \qrcode[height=2cm]{#2}
  \end{minipage}
  \vspace{1em}
}

% --- IDEA 4: "Museum Plaque" ---
\newcommand{\symbolcardD}[4]{%
  \noindent\hrule\vspace{1em}
  \begin{minipage}{\textwidth}
    \centering
    \includegraphics[width=7cm]{#1}
  \end{minipage}
  \vspace{1em}
  
  \begin{minipage}[t]{0.8\textwidth}
    \raggedright
    {\Large\bfseries #3}\par\medskip
    #4
  \end{minipage}%
  \hfill
  \begin{minipage}[t]{0.18\textwidth}
    \raggedleft
    \qrcode[height=1.5cm]{#2}
  \end{minipage}
  \vspace{1.5em}
}

% --- IDEA 5: "Asymmetric Balance" ---
\newcommand{\symbolcardE}[4]{%
  \noindent\hrule\vspace{1.5em}
  \begin{minipage}[t]{0.4\textwidth}
    \includegraphics[width=\linewidth, keepaspectratio]{#1}
  \end{minipage}\hfill
  \begin{minipage}[t]{0.55\textwidth}
    {\Large\bfseries #3}\par\medskip
    #4
    \par\vfill % Pushes QR code to the bottom
    \raggedleft
    \qrcode[height=2cm]{#2}
  \end{minipage}
  \vspace{1.5em}
}

% --- IDEA 5.1: "Centered Asymmetric Balance" ---
\newcommand{\symbolcardF}[4]{%
  \noindent\hrule\vspace{1.5em}
  % We use [c] to vertically center the image and text blocks relative to each other.
  \begin{minipage}[c]{0.4\textwidth}
    \includegraphics[width=\linewidth, keepaspectratio]{#1}
  \end{minipage}\hfill
  \begin{minipage}[c]{0.55\textwidth}
    {\Large\bfseries #3}\par\medskip
    #4
    \par\vfill % Pushes QR code to the bottom of this block
    \raggedleft
    \qrcode[height=2cm]{#2}
  \end{minipage}
  \vspace{1.5em}
}

% --- IDEA 5.2: "Text-Favored Asymmetric Balance" ---
\newcommand{\symbolcardG}[4]{%
  \noindent\hrule\vspace{1.5em}
  % Image column is narrower (0.3), text column is wider (0.65)
  \begin{minipage}[c]{0.3\textwidth}
    \includegraphics[width=\linewidth, keepaspectratio]{#1}
  \end{minipage}\hfill
  \begin{minipage}[c]{0.65\textwidth}
    {\Large\bfseries #3}\par\medskip
    #4
    \par\vfill % Pushes QR code to the bottom of this block
    \raggedleft
    \qrcode[height=2cm]{#2}
  \end{minipage}
  \vspace{1.5em}
}

\begin{document}

% \section*{Layout Idea 1: Three-Column}
% \symbolcardA{HEP/ALICE_dEdx.png}{https://alice-collaboration.web.cern.ch/}{ALICE $dE/dx$}{Specific energy loss measurement (Bethe--Bloch like curve) used for particle identification in the ALICE TPC at the LHC.}

% \section*{Layout Idea 2: Image-Centric}
% \symbolcardB{HEP/ALICE_dEdx.png}{https://alice-collaboration.web.cern.ch/}{ALICE $dE/dx$}{Specific energy loss measurement (Bethe--Bloch like curve) used for particle identification in the ALICE TPC at the LHC.}

% \section*{Layout Idea 3: Refined Two-Column}
% \symbolcardC{HEP/ALICE_dEdx.png}{https://alice-collaboration.web.cern.ch/}{ALICE $dE/dx$}{Specific energy loss measurement (Bethe--Bloch like curve) used for particle identification in the ALICE TPC at the LHC.}

% \section*{Layout Idea 4: The "Museum Plaque"}
% \symbolcardD{HEP/ALICE_dEdx.png}{https://alice-collaboration.web.cern.ch/}{ALICE $dE/dx$}{Specific energy loss measurement (Bethe--Bloch like curve) used for particle identification in the ALICE TPC at the LHC.}

\section*{Layout Idea 5: The "Asymmetric Balance"}
\symbolcardE{HEP/ALICE_dEdx.png}{https://alice-collaboration.web.cern.ch/}{ALICE $dE/dx$}{Specific energy loss measurement (Bethe--Bloch like curve) used for particle identification in the ALICE TPC at the LHC.}

\section*{Layout Idea 5.1: Centered Asymmetric Balance}
\symbolcardF{HEP/ALICE_dEdx.png}{https://alice-collaboration.web.cern.ch/}{ALICE $dE/dx$}{Specific energy loss measurement (Bethe--Bloch like curve) used for particle identification in the ALICE TPC at the LHC.}

\symbolcardF{HEP/dirac_equation.png}{https://en.wikipedia.org/wiki/Dirac_equation}{Dirac Equation}{Relativistic wave equation for spin-$1/2$ fermions predicting antimatter. $(i\hbar\gamma^\mu \partial_\mu - mc)\psi = 0$}

\symbolcardF{HEP/AMBER_logo.png}{https://home.cern/science/experiments/amber}{AMBER Logo}{Logo of the AMBER fixed-target experiment at CERN (successor to COMPASS) studying hadron structure and QCD dynamics.}

\section*{Layout Idea 5.2: Text-Favored Asymmetric Balance}
\symbolcardG{HEP/ALICE_dEdx.png}{https://alice-collaboration.web.cern.ch/}{ALICE $dE/dx$}{Specific energy loss measurement (Bethe--Bloch like curve) used for particle identification in the ALICE TPC at the LHC.}

\symbolcardG{HEP/dirac_equation.png}{https://en.wikipedia.org/wiki/Dirac_equation}{Dirac Equation}{Relativistic wave equation for spin-$1/2$ fermions predicting antimatter. $(i\hbar\gamma^\mu \partial_\mu - mc)\psi = 0$}

\symbolcardG{HEP/AMBER_logo.png}{https://home.cern/science/experiments/amber}{AMBER Logo}{Logo of the AMBER fixed-target experiment at CERN (successor to COMPASS) studying hadron structure and QCD dynamics.}

\noindent\hrule

\end{document}
