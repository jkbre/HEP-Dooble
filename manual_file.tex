% Symbol Guide for Doodles Cards (v1)
% Auto-generated first draft. Edit for accuracy/refinement.
\documentclass[11pt,a4paper]{article}
\usepackage[margin=1.9cm]{geometry}
\usepackage[utf8]{inputenc}
\usepackage[T1]{fontenc}
\usepackage{graphicx}
% \usepackage{wrapfig} % Used by the original symbol card command
\usepackage{amsmath,amssymb}
\usepackage{enumitem}
\usepackage{hyperref}
\usepackage{qrcode}
\usepackage{xcolor}
% \usepackage{xparse}
\usepackage{fancyhdr}
\setlist[itemize]{leftmargin=*,nosep,topsep=2pt,partopsep=0pt}
\hypersetup{colorlinks=true,linkcolor=blue,urlcolor=teal}
\pagestyle{fancy}
\fancyhf{} % Clear all header and footer fields
% \fancyfoot[C]{\small Draft generated on \today\ -- Review for accuracy before publication.}
\fancyfoot[C]{\small Guide generated on \today.}
\fancyfoot[R]{\thepage} % Page number on the right
\renewcommand{\headrulewidth}{0pt} % Remove header rule
\renewcommand{\footrulewidth}{0pt} % Remove footer rule
\setlength{\parindent}{0pt}

% \newcommand{\symbolcard}[4]{%
%   \includegraphics[width=5cm]{#1}%
%
%   \begin{wrapfigure}{r}{0.2\textwidth}%
%     \qrcode{#2}%
%   \end{wrapfigure}%
%   {\Large \textbf{#3}} - #4%
%
%   \vspace{3\baselineskip}%
% }

% Custom command for creating symbol cards with images, descriptions, and QR codes
% Usage: \symbolcard{image_path}{url}{title}{description}
% Parameters:
%   #1: Path to the image file (relative to document)
%   #2: URL for the QR code (typically a reference link)
%   #3: Title/name of the symbol or concept
%   #4: Detailed description or explanation text
% Layout: Two-column design with image on left (30% width) and content on right (65% width)
% Features: Horizontal rule separator, responsive image scaling, bottom-aligned QR code
\newcommand{\symbolcard}[4]{%
  \noindent\hrule\vspace{18pt}                                % Top horizontal line separator + spacing
  % Image column is narrower (0.3), text column is wider (0.65)
  \begin{minipage}[c]{0.3\textwidth}                           % Left column: 30% width for image
    \includegraphics[width=\linewidth, keepaspectratio]{#1}    % The main image/symbol/logo
  \end{minipage}\hfill                                         % Gap between columns
  \begin{minipage}[c]{0.65\textwidth}                          % Right column: 65% width for text content
    {\Large\bfseries #3}\par\medskip                           % Large bold title at top of right column
    #4                                                         % Main description text body
    \par\vfill                                                 % Pushes QR code to bottom of right column
    \vspace{12pt}                                              % Horizontal gap between description and QR code
    \raggedleft                                                % Right-align the QR code
    \qrcode[height=2cm]{#2}                                    % Small QR code at bottom-right
  \end{minipage}
  \vspace{18pt}                                               % Bottom spacing before next card
}

\begin{document}
\begin{center}\Large\bfseries Particle / HEP Symbol \& Logo Guide (Draft)\end{center}\symbolcard{HEP/ALICE_dEdx.png}{https://en.wikipedia.org/wiki/Bethe\_formula?useskin=vector}{ALICE $dE/dx$}{Bethe-Bloch curve from the ALICE experiment at CERN, illustrating the principle of particle identification (PID) through specific energy loss (dE/dx) measurements in a detector.}
\symbolcard{HEP/AMBER_logo.png}{https://home.cern/news/news/physics/meet-amber}{AMBER Logo}{Logo of the AMBER fixed-target experiment at CERN (successor to COMPASS) studying hadron structure and QCD dynamics.}
\symbolcard{HEP/anihilation_diagram.png}{https://en.wikipedia.org/wiki/Electron\%E2\%80\%93positron\_annihilation}{Annihilation Diagram}{Tree-level electron-positron annihilation into a muon pair via virtual photon (or Z at higher energies). $e^+ e^- \to \gamma^*/Z^0 \to \mu^+ \mu^-$}
\symbolcard{HEP/baryon_decuplet.png}{https://en.wikipedia.org/wiki/Eightfold\_way\_(physics)\#Baryons}{Baryon Decuplet}{Spin-$3/2$ ground-state baryon SU(3) flavor decuplet ($\Delta$, $\Sigma^*$, $\Xi^*$, $\Omega$).}
\symbolcard{HEP/baryon_octet.png}{https://en.wikipedia.org/wiki/Eightfold\_way\_(physics)\#Baryons}{Baryon Octet}{Spin-$1/2$ ground-state baryon SU(3) flavor octet (N, $\Lambda$, $\Sigma$, $\Xi$).}
\symbolcard{HEP/beta_decay_diagram.png}{https://en.wikipedia.org/wiki/Beta\_decay}{Beta Decay Diagram}{Weak interaction $d \to u$ quark transition emitting $W^-$ leading to $e^-$. Also called a beta decay ($n \to p + e^- + \bar\nu_e$). and $\bar{\nu}_e$ (neutron beta decay). $n \to p + e^- + \bar\nu_e$}
\symbolcard{HEP/box_diagram.png}{https://en.wikipedia.org/wiki/Neutral\_particle\_oscillation}{Box Diagram}{Loop-level box diagram contributing to neutral meson mixing (e.g. $B^0$-$\bar{B}^0$ - matter-antimatteroscillations).}
\symbolcard{HEP/bremsstrahlung.png}{https://en.wikipedia.org/wiki/Bremsstrahlung}{Bremsstrahlung}{Radiative emission from charged particle accelerated in a Coulomb field.}
\symbolcard{HEP/bubble_chamber_omega.png}{https://en.wikipedia.org/wiki/Omega\_baryon\#Discovery}{Bubble Chamber $\Omega^-$}{Discovery image of $\Omega^-$ (sss) in bubble chamber confirming SU(3) decuplet.}
\symbolcard{HEP/cern_logo.png}{https://en.wikipedia.org/wiki/CERN}{CERN Logo}{Logo of CERN, the European Organization for Nuclear Research.}
\symbolcard{HEP/cherenkov.png}{https://en.wikipedia.org/wiki/Cherenkov\_radiation}{Cherenkov Radiation}{Emission of coherent light when a charged particle exceeds phase velocity of light in a medium.}
\symbolcard{HEP/CKM.png}{https://en.wikipedia.org/wiki/Cabibbo\%E2\%80\%93Kobayashi\%E2\%80\%93Maskawa\_matrix}{CKM Matrix}{Quark flavor mixing matrix in charged-current weak interactions.}
\symbolcard{HEP/clebsh_gordan.png}{https://en.wikipedia.org/wiki/Clebsch\%E2\%80\%93Gordan\_coefficients}{Clebsch-Gordan Coefficients}{Coupling coefficients for adding angular momenta in quantum mechanics. $ |JM\rangle = \sum_{jm} |j_1 m_1; j_2 m_2\rangle \langle j_1 j_2 m_1 m_2 | JM \rangle$}
\symbolcard{HEP/cms_logo.png}{https://en.wikipedia.org/w/index.php?title=Compact\_Muon\_Solenoid\&useskin=vector}{CMS Logo}{Logo of the Compact Muon Solenoid detector at the LHC.}
\symbolcard{HEP/compass_logo.png}{https://en.wikipedia.org/wiki/COMPASS\_experiment?useskin=vector}{COMPASS Logo}{Logo of the COMPASS fixed-target experiment at CERN SPS.}
\symbolcard{HEP/cyclotron.png}{https://en.wikipedia.org/wiki/Cyclotron}{Cyclotron}{Scheme of an early circular accelerator using magnetic field and alternating electric field. Invented by Ernest Lawrence in 1929-1930 at the University of California, Berkeley.}
\symbolcard{HEP/dalitz_plot.png}{https://en.wikipedia.org/wiki/Dalitz\_plot}{Dalitz Plot}{Three-body decay event density plot in invariant mass-squared plane.}
\symbolcard{HEP/desy_logo.png}{https://en.wikipedia.org/wiki/DESY}{DESY Logo}{Logo of DESY (Deutsches Elektronen-Synchrotron), German accelerator center.}
\symbolcard{HEP/dirac_equation.png}{https://en.wikipedia.org/wiki/Dirac\_equation}{Dirac Equation}{Relativistic wave equation for spin-$1/2$ fermions that predicted antimatter.}
\symbolcard{HEP/dis_diagram.png}{https://en.wikipedia.org/wiki/Deep\_inelastic\_scattering}{DIS Diagram}{Deep inelastic scattering of lepton on nucleon probing partons.}
\symbolcard{HEP/drell-yan_diagram.png}{https://en.wikipedia.org/wiki/Drell\%E2\%80\%93Yan\_process}{Drell-Yan Diagram}{Quark-antiquark annihilation producing lepton pair. $q \bar q \to \gamma^*/Z^0 \to \ell^+\ell^-$}
\symbolcard{HEP/dune_logo.png}{https://en.wikipedia.org/wiki/Deep\_Underground\_Neutrino\_Experiment}{DUNE Logo}{Logo of the Deep Underground Neutrino Experiment.}
\symbolcard{HEP/em_cascade.png}{https://en.wikipedia.org/wiki/Particle\_shower\#Electromagnetic\_showers}{EM Cascade}{Electromagnetic shower from repeated pair production and bremsstrahlung.  Length of a cascade is proportional to an energy ratio $X_{max} \approx X_0 \frac{\ln(E_0/E_c)}{\ln 2}$}
\symbolcard{HEP/fermi_golden_rule.png}{https://en.wikipedia.org/wiki/Fermi\%27s\_golden\_rule}{Fermi Golden Rule}{Transition rate from perturbation in quantum mechanics.}
\symbolcard{HEP/fuw_logo.png}{https://www.fuw.edu.pl/en}{FUW Logo}{Logo of Faculty of Physics, University of Warsaw.}
\symbolcard{HEP/gem.png}{https://en.wikipedia.org/wiki/Gas\_electron\_multiplier}{GEM Detector}{Gas Electron Multiplier micro-pattern gaseous detector.}
\symbolcard{HEP/ggH_diagram.png}{https://en.wikipedia.org/wiki/Higgs\_boson\#Production}{ggH Diagram}{Gluon fusion loop (top quark) producing a Higgs boson.}
\symbolcard{HEP/grand_logo.png}{https://grand.cnrs.fr/overview/}{GRAND Logo}{Logo of the Giant Radio Array for Neutrino Detection.}
\symbolcard{HEP/higgs-potential.png}{https://en.wikipedia.org/wiki/Higgs\_mechanism}{Higgs Potential}{Mexican-hat potential leading to electroweak symmetry breaking. $V(\phi) = -\mu^2\phi^\dagger\phi + \lambda (\phi^\dagger\phi)^2$.}
\symbolcard{HEP/hypernucleus.png}{https://en.wikipedia.org/wiki/Hypernucleus}{Hypernucleus}{Nucleus containing one or more hyperons (e.g., $\Lambda$). Hyperons are a category of baryon particles that carry non-zero strangeness quantum number, which is conserved by the strong and electromagnetic interactions.}
\symbolcard{HEP/hypothesis_test.png}{https://en.wikipedia.org/wiki/Statistical\_significance\#In\_particle\_physics}{Hypothesis Test}{Statistical decision framework widely use in Physics, especially in High Energy Physics (e.g., discovery threshold is $5\sigma$).}
\symbolcard{HEP/kamiokande_logo.png}{https://en.wikipedia.org/w/index.php?title=Super-Kamiokande\&useskin=vector}{Kamiokande Logo}{Logo representing Kamiokande / Super-Kamiokande neutrino detector.}
\symbolcard{HEP/kaon_mixing.png}{https://en.wikipedia.org/wiki/Neutral\_kaon\_mixing}{Kaon Mixing}{Neutral kaon oscillation via weak interactions.}
\symbolcard{HEP/lagrangian.png}{https://en.wikipedia.org/wiki/Lagrangian\_(field\_theory)}{Lagrangian Density}{Generic form describing the Lagrangian Density of the Standard Model of Paricles.}
\symbolcard{HEP/lorentz_tr_matrix.png}{https://en.wikipedia.org/wiki/Lorentz\_transformation}{Lorentz Transformation Matrix}{Matrix representing boost/rotation in Minkowski spacetime.}
\symbolcard{HEP/luxe_logo.png}{https://luxe.desy.de/}{LUXE Logo}{Logo of the LUXE (Laser Und XFEL Experiment) probing QED in strong-field.}
\symbolcard{HEP/m4l_Higgs.png}{https://en.wikipedia.org/wiki/File:Higgs4Lepton.png}{Higgs production channel (m4l)}{Invariant mass distribution for Higgs to four-lepton decay ($pp \to H \to ZZ^{(*)} \to 4\ell$), the "golden channel" for its discovery.}
\symbolcard{HEP/nu_oscillations.png}{https://en.wikipedia.org/wiki/Neutrino\_oscillation}{Neutrino Oscillations}{Flavor change due to mixing and mass differences (with probability approximation of $P_{\alpha\to\beta} \approx \sin^2(2\theta) \sin^2\left(\frac{\Delta m^2 L}{4E}\right)$).}
\symbolcard{HEP/pair_production.png}{https://en.wikipedia.org/wiki/Pair\_production}{Pair Production}{Photon conversion into $e^+ e^-$ in external Coulomb field, for field energy above a critical value ($E_\gamma > 2 m_e c^2$).}
\symbolcard{HEP/penguin_diagram.png}{https://en.wikipedia.org/wiki/Penguin\_diagram}{Penguin Diagram}{Loop diagram inducing flavor-changing neutral currents ($b \to s \gamma$, $b \to s \ell^+\ell^-$).}
\symbolcard{HEP/photomultiplier.png}{https://en.wikipedia.org/wiki/Photomultiplier\_tube}{Photomultiplier Tube}{Vacuum tube amplifying photoelectrons via dynode chain.}
\symbolcard{HEP/PMNS.png}{https://en.wikipedia.org/wiki/Pontecorvo\%E2\%80\%93Maki\%E2\%80\%93Nakagawa\%E2\%80\%93Sakata\_matrix}{PMNS Matrix}{Neutrino flavor mixing matrix (Pontecorvo-Maki-Nakagawa-Sakata). Neutrino oscillations parametrization.}
\symbolcard{HEP/pretzelosity_TMDs.png}{https://arxiv.org/abs/0807.2524}{Pretzelosity TMD}{Transverse momentum dependent distribution related to nucleon spin-orbit correlations.}
\symbolcard{HEP/quadrupole.png}{https://en.wikipedia.org/wiki/Quadrupole\_magnet}{Quadrupole Magnet}{Magnet focusing charged particle beams with linear field gradient. Used in accelerators like LEP and LHC.}
\symbolcard{HEP/quantum_spin.png}{https://en.wikipedia.org/wiki/Spin\_(physics)}{Quantum Spin}{Intrinsic angular momentum of a quantum particle.}
\symbolcard{HEP/schwinger_diagram.png}{https://en.wikipedia.org/wiki/Schwinger\_effect}{Schwinger Diagram}{Vacuum pair production in a strong electric field (Schwinger effect).}
\symbolcard{HEP/scintillator.png}{https://en.wikipedia.org/wiki/Scintillator}{Scintillator}{Material emitting photons when traversed by ionizing radiation.}
\symbolcard{HEP/shine_logo.png}{https://en.wikipedia.org/wiki/NA61\_experiment}{SHINE Logo}{Logo of SHINE (NA61) experiment at CERN SPS.}
\symbolcard{HEP/silicon_detector.png}{https://en.wikipedia.org/wiki/Semiconductor\_detector}{Silicon Detector}{Semiconductor tracking detector using depleted silicon.}
\symbolcard{HEP/sk_event.png}{https://t2k-experiment.org/super-kamiokande-event-displays/}{Super-K Event}{Cherenkov light pattern from neutrino interaction in Super-Kamiokande.}
\symbolcard{HEP/unitarity_triangle.png}{https://en.wikipedia.org/wiki/Unitarity\_triangle}{Unitarity Triangle}{Graphical representation of CKM unitarity relation ($V_{ud}V_{ub}^* + V_{cd}V_{cb}^* + V_{td}V_{tb}^* = 0$).}
\symbolcard{HEP/valence_q_pdf.png}{https://en.wikipedia.org/wiki/Parton\_distribution\_function}{Valence Quark PDFs}{Momentum distributions of valence quarks inside nucleon.}
\symbolcard{HEP/warsawtpc_logo.png}{https://inspirehep.net/literature/2736035}{Warsaw TPC Logo}{Logo of Warsaw TPC project / group involvement.}
\symbolcard{HEP/WMAP_2012.png}{https://en.wikipedia.org/wiki/Wilkinson\_Microwave\_Anisotropy\_Probe}{WMAP 2012}{CMB anisotropy map from WMAP mission (2012 release).}
\symbolcard{HEP/Wu_experiment.png}{https://en.wikipedia.org/wiki/Wu\_experiment}{Wu Experiment}{Parity violation in $\beta$ decay observed in 1957 cobalt-60 experiment.}
\symbolcard{HEP/xfel_cavity.png}{https://www.xfel.eu/facility/overview/how\_it\_works/index\_eng.html}{XFEL Cavity}{Superconducting RF accelerating cavity for X-ray free-electron laser (undulator).}
\symbolcard{HEP/z_peak.png}{https://www.researchgate.net/figure/Average-LEP-hadronic-cross-sections-as-a-function-of-centre-of-mass-energy-The-errors\_fig2\_2004159}{Z Peak}{Resonant enhancement in $e^+e^- \to \text{hadrons}$ near $Z^0$ mass.}
\end{document}